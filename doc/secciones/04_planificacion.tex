\chapter{Planificación}

\section{Metodología utilizada}


\section{Temporización}

\section{Usuarios o partes interesadas}
Antes de definir las historias de usuario, se han analizado los usuarios o partes interesadas en el proyecto, para enfocar el proyecto en resolver sus problemas. Son los siguientes:
\begin{itemize}
    \item \textbf{Tribunal}: el tribunal es el encargado de valorar este trabajo de fin de máster y, por tanto, busca que tenga una memoria bien redactada con todo el proceso de estudio, análisis y desarrollo del proyecto.
    \item \textbf{Investigador}: el investigador, o analista de datos, necesita tener la información estructurada de los tropos para construir modelos que le permitan hacer inferencias sobre las relaciones que tienen los tropos entre sí y llegar a conclusiones que sirvan para su investigación.
\end{itemize}

\section{Historias de usuario}
Sabiendo el problema que se quiere resolver, y las partes interesadas, se pueden definir las historias de usuario que guiarán todo el desarrollo de principio a fin, guiando las decisiones que se irán tomando para llegar a cumplir los objetivos.

\subsection{[HU01] Investigador - Obtener información estructurada sobre todos los tropos}
Como investigador necesito poder obtener información estructurada sobre todos los tropos existentes de cualquier tipo de medio audiovisual, en un formato estandarizado y con una estructura definida para que sean de fácil acceso tanto a humanos como programas para su análisis.

\subsection{[HU02] Investigador - Obtener un subconjunto de la información}
Como investigador necesito poder elegir qué parte de la información sobre tropos quiero obtener, ya sea de un medio audiovisual concreto, todo un género, etc.

\subsection{[HU03] Investigador - Relacionar medios audiovisuales con su ficha en IMDB}
Como investigador necesito obtener información de metadatos sobre cualquier medio audiovisual, para lo cual necesito que los medios audiovisuales de los cuales se han sacado sus tropos se puedan relacionar con su ficha en IMDB.

\section{Seguimiento del desarrollo}
