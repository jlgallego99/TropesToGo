\chapter{Introducción}

Un tropo es una figura retórica o convención que se usa en la creación de historias y es reconocible por la audiencia. Un tropo puede ser una situación, un tipo de personaje, una estructura narrativa o cualquier patrón que se utilice a la hora de contar una historia, de forma que no la interrumpa (en cuyo caso, sería un cliché) sino que ayuda a entenderla y construirla mejor. Los tropos tienen interés como una parte muy importante del puzle que es producir una obra de cualquier tipo, ya sea una serie, película, videojuego, libro... 

Estos se pueden analizar para obtener una serie de pautas que puedan servir para crear nuevas obras más efectivas, ya sea analizando cómo evoluciona el uso de tropos a lo largo del tiempo y cuáles son los más adecuados para representar un tipo de género o historia concreta. En resumen, los tropos son clave en la popularidad que puede llegar a tener una obra. 

TvTropes\footnote{\url{https://tvtropes.org/}} es una wiki que describe y almacena ejemplos de tropos en cualquier tipo de medio audiovisual. Contiene una gran lista de tropos, con una descripción y ejemplos reales de dónde y cómo se aplican. TvTropes indica para cada entrada de una obra audiovisual todos los tropos que se le han identificado y por qué, con ejemplos, siendo esta la principal fuente de información que proporciona la web. Esta información está constantemente creciendo.

Sin embargo, tal y como se define en la propia web, el objetivo de TvTropes es ser una wiki sobre tropos que se usan para contar historias y mostrarlos de un modo accesible y divertido de leer, destinado al usuario final. Esta información no está estructurada y, por tanto, no existe una manera de acceder a ella (mediante una API, por ejemplo) para su uso en ciencia de datos o inteligencia artificial. Además de esto, TvTropes es una web centrada en los tropos y nada más que los tropos, por lo que no manejan información secundaria sobre los contenidos audiovisuales que tienen, los cuales pueden ser también interesantes de analizar.

En estos análisis entran trabajos de investigación como el de \cite{garcia2020tropes} o el de \cite{garcia2021simpsons}, los cuales actúan como principal motivación de este proyecto, puesto que ambos utilizan tanto una base de datos existente que está desactualizada como una biblioteca de scraping de TvTropes que no consigue extraer toda la información que tiene la web actualmente, la cual ha cambiado bastante con el tiempo.

Por tanto, ante la necesidad de tener toda la información completa de TvTropes, se quiere construir un scraper capaz de obtener de forma estructurada esta información sobre tropos, ampliando aún más su funcionalidad para obtener tropos de cualquier medio audiovisual e información de metadatos adicional, para que cualquier programador, científico de datos o investigador pueda emplearla para crear modelos, programas o investigaciones.\\

Este trabajo se divide en tres partes diferenciadas. La primera empezará describiendo el problema que se desea resolver y los objetivos que se quieren alcanzar, para luego analizar las alternativas que existen actualmente en materia de scrapers y herramientas que extraen la información de la web de TvTropes, así como la situación actual de la propia página y los retos que presenta a la hora de extraer su información. La segunda parte se centrará en explicar y desarrollar la metodología de desarrollo ágil que se ha utilizado y cómo se ha guiado todo el desarrollo del software hasta el final. Por último, la tercera parte desarrollará todo el trabajo realizado y cómo se ha llegado a la solución final, valorando finalmente si se han alcanzado los objetivos propuestos.\\

Este proyecto es software libre, está liberado con la licencia GPLv3\footnote{\url{http://www.gnu.org/licenses/gpl.html}} y se puede encontrar en un repositorio público de GitHub\footnote{\url{https://github.com/jlgallego99/TropesToGo}}.
