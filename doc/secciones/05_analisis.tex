\chapter{Análisis del problema}
En este capítulo se modela el problema de construcción de un \textit{scraper} de
TvTropes aplicando el patrón de \textit{Domain Driven Design}. Luego, se analiza
la estructura que sigue la web, identificando los puntos más importantes para
saber cómo abordar la creación de la araña que explore todas las páginas
relevantes de TvTropes y la extracción de la información que queremos en cada
una de ellas. El objetivo general del capítulo es servir de inicio para saber
cómo abordar la solución al problema y así poder llevarlo luego a lo práctico en
la implementación del \textit{software} en el capítulo 6, en el que se
desarrollará la estrategia final para resolver el problema de la extracción de
datos en TvTropes.

\section{Modelado del problema}

\textit{Domain Driven Design}

\section{Análisis de TvTropes}
Un wiki complejo como TvTropes, en el que tanto la estructura concreta de una
página como entre ellas cambia bastante, contiene información que es de difícil
acceso y está desperdigada por distintos lugares. A veces, no existe siquiera.
Es por esto que a continuación, tras haber visto cómo se pretende modelar el
problema y organizar el código, se estudian los problemas y particularidades que
tiene TvTropes y que queremos resolver con el \textit{scraper} para poder
extraer toda la información que sea posible de la manera más sencilla y
eficiente. Para ello, se estudia la organización de las páginas de TvTropes
tanto entre sí como dentro de ellas, y se observan los puntos más importantes
que el bot debe saber identificar para poder desarrollar una estrategia en el
siguiente capítulo conociendo todos los puntos relevantes de la web.

Primero, se hará una breve introducción sobre cómo funciona TvTropes en general.
Luego, se analizará la organización entre páginas para el desarrollo del
\textit{crawler} o araña. Finalmente, se analizará la estructura del código HTML
dentro de una misma página, para el desarrollo del \textit{scraper}.

\subsection{Vista general de TvTropes}
Antes de pasar a entender en profundidad cómo están relacionadas las partes de
TvTropes, es necesario primero ver cómo funciona la web en general y qué
contenidos ofrece.

En este trabajo el interés principal está en identificar y extraer los
\textit{tropos}, que son la principal fuente de información de TvTropes,
conocidos en inglés y en la propia web como \textit{tropes}. Pero además,
TvTropes trabaja con otra información que también se puede considerar como
principal y a la que se accede desde la barra de navegación superior de la
página. Para el ámbito de este trabajo, el más importante es el \textit{Media},
que son los trabajos que existen en un medio audiovisual concreto y que
contienen unos \textit{tropos} asociados. Existe otra sección, la de índices,
que es muy relevante para el estudio de la web en este capítulo y qué se verá en
la siguiente sub sección. Además de esto están otros que no consideramos para el
\textit{scraper}, como el foro o los vídeos.

Una página que describe un \textit{tropo} suele presentar, generalmente, una
descripción de cómo funciona, ejemplos de obras que los tienen organizados por
distintos índices y una lista de sub \textit{tropos} relacionados con él. Que en
una obra aparezca un \textit{tropo} no significa que necesariamente aparezcan
los relacionados con él. Como ejemplo está
\begin{otherlanguage}{english}\textit{TheChosenOne}\end{otherlanguage}, que
presenta como uno de sus sub \textit{tropos}
\begin{otherlanguage}{english}\textit{The Antichrist}\end{otherlanguage}, sin
embargo, no todas las obras en las que aparezca un personaje que es el elegido
tiene por qué ser el anticristo.

Por otro lado, las páginas de obras, aunque visualmente parecidas, presentan
otra información. Al igual que con los \textit{tropos}, lo primero que tienen es
el título y un resumen con información general de la obra, \textit{tropos}
referenciados, enlaces a vídeos, y muchas otras que varía muchísimo entre
páginas. Además de esto, está la parte más importante: una lista de los
\textit{tropos} principales que la comunidad ha identificado, con su
correspondiente descripción, y generalmente presentados de múltiples formas que
dificultan su extracción y que se analizará más en profundidad a lo largo de
este capítulo. Otro aspecto muy destacable en una página de una obra es que
contiene múltiples sub páginas que presentan nuevos \textit{tropos} de ese
trabajo desde otros puntos de vista, personajes, curiosidades, citas, etc.

Además de las secciones vistas, en la barra de navegación superior existe una
barra de búsqueda en la que buscar por nombre el \textit{tropo} u obra
audiovisual en la que esté interesado el usuario y una sección para gestionar
una cuenta de usuario. Las cuentas de usuario no son necesarias para consultar
TvTropes; se emplean para contribuir a la comunidad tanto en el foro como en la
creación de nuevos \textit{tropos}, por lo que se salen fuera del ámbito de este
trabajo y el \textit{scraper} no las tiene que tener en cuenta. Esto además
facilita ciertos aspectos legales, ya que, al tener toda la información
disponible públicamente y sin tener que pasar por la creación de una cuenta que
tenga una serie de condiciones asociadas, permite que el bot pueda recoger esa
información sin mayores consideraciones.

\subsection{El indexado en TvTropes}
Para construir una buena araña que sea capaz de explorar todas las páginas de
TvTropes es necesario entender cómo están relacionadas entre sí las páginas.
Para esto es necesario conocer cómo funciona el indexado en TvTropes, es decir,
de qué manera están organizadas todas las secciones de la web y cómo un usuario
puede encontrar lo que quiere. Al saber cómo un usuario puede encontrar lo que
quiere, también sabemos cómo debe comportarse el \textit{crawler} para indexar
las páginas que necesitamos.

La araña necesita un buen punto de partida a partir del cual, indexando todos
los hipervínculos que encuentre, pueda ir explorando recursivamente y encontrar
todas las páginas que necesita luego el \textit{scraper}. Todos los
\textit{tropos} están listados en un índice llamado \textit{Main}. En él se
puede ver que todos los \textit{tropos} siguen en su URL la misma forma que
consiste en \texttt{https://tvtropes.org/pmwiki/pmwiki.php/Main/} más el nombre
del tropo en estilo \textit{CamelCase}. Sin embargo, existen páginas de este
estilo que no son tropos, sino que son nuevos índices que sí que desembocan en
\textit{tropos} finalmente. Además, el saber la URL no sirve para encontrarlos,
antes se necesita saber cuáles existen y sus nombres.

Existe dos índices de \textit{tropos} que los ordenan alfabéticamente. Uno de
esos índices\footnote{\url{https://tvtropes.org/pmwiki/index_namespaces.php}} da
una barra de búsqueda en la que poder escribir el nombre del tropo que se busca
y una lista que, sin embargo, no contiene ni mucho menos todos los
\textit{tropos}. Al buscar, por ejemplo, aquellos que empiecen por la letra Z
vemos que únicamente se listan tres, mientras que con una breve búsqueda en
cualquier obra se puede ver que hay muchísimos otros con la letra Z que no están
listados. Esto evidencia que en este índice faltan muchos \textit{tropos} y no
puede usarse como un buen punto de partida para ellos. Por suerte, existe otro
índice que sí que los contiene
todos,\footnote{\url{https://tvtropes.org/pmwiki/pagelist_having_pagetype_in_namespace.php?n=Main&t=trope}}
ya que, es un script interno de TvTropes que hace una búsqueda en toda la web
con páginas de un tipo llamado \textit{trope}. Por tanto, se debería tener en
cuenta este último porque nos asegura que están listados todos los existentes en
la página y además es el que se recomienda en los
foros\footnote{\url{https://tvtropes.org/pmwiki/posts.php?discussion=14420393930A40584600&page=1}}.

Por último, se puede observar que la URL de este último índice contiene
parámetros que, si se modifican, nos permiten obtener un índice completo de, por
ejemplo, todas las
películas\footnote{\url{https://tvtropes.org/pmwiki/pagelist_having_pagetype_in_namespace.php?n=Film&t=work}}
o todos los
videojuegos\footnote{\url{https://tvtropes.org/pmwiki/pagelist_having_pagetype_in_namespace.php?n=Videogame&t=work}}.

Dirigir la búsqueda de las obras desde los \textit{tropos} no es eficaz ni
correcto, debido a que, la sección de ejemplos de obras dentro de la página de
un \textit{tropo} presenta solo los casos más relevantes y rara vez todas las
ocurrencias existentes. Por tanto, tiene más sentido encontrarlos desde la
propia página de la obra, que será más profunda y completa. La existencia de
unos índices completos donde encontrar todas las páginas de cualquier tipo
permite que la exploración dentro de una página concreta sea únicamente para
extraer información y no hipervínculos para que la araña indexe, así que entra
en el dominio del \textit{scraper}. Además de esto, generalmente las páginas de
\textit{tropos} y obras están constantemente referenciándose mutuamente, y eso
llevaría a bucles difícilmente controlables.

Estos índices también permiten evitar errores observables en el repositorio de
\textit{Tropescraper}, en el que algunos usuarios reportan que existen páginas
que el \textit{scraper} considera como tropos, pero que no lo son. Un ejemplo de
esto es \begin{otherlanguage}{english}\textit{Films of the
2020s}\end{otherlanguage}\footnote{\url{https://tvtropes.org/pmwiki/pmwiki.php/Main/FilmsOfThe2020s}},
en el cual se listan películas a partir del año 2020 y no es un \textit{tropo}
pese a colgar del índice \textit{Main}.

Hasta ahora se ha hablado principalmente del índice de \textit{tropos}, y se ha
visto que se puede utilizar para obtener también un índice con todas las
películas o todos los videojuegos. Sin embargo, estos son solo dos de los tipos
de obras más reconocibles, lo que en TvTropes se conoce como
\textit{media}\footnote{\url{https://tvtropes.org/pmwiki/pmwiki.php/Main/Media}}.
Este índice es bastante complejo porque TvTropes hace una gran separación entre
los trabajos u obras según el medio en el que se presentan. Por ejemplo, en la
categoría de animación distingue entre animación occidental, animación asiática
o anime japonés, o en la categoría de películas distingue entre películas de
acción real o películas de animación. Tropescraper extraía únicamente todas las
películas del índice \textit{Film}, lo que llevaba a dejarse muchas obras que
también son películas, como puede ser El Viaje de Chihiro, que está en la
sección de anime. En este trabajo se usarán las categorías de \textit{media} tal
y como se definen en TvTropes al ser imposible diferenciar, por ejemplo, una
serie de anime de una película anime al estar las dos bajo la misma sección de
anime. En general, estas categorías son subjetivas, por lo que no hay una
solución concreta, sin embargo, se busca poder extraer la información de todos
los tipos de \textit{media} posibles.

En resumen, se han podido encontrar índices que sabemos con toda seguridad que
contienen todos los tropos y obras separadas por tipo, por lo que esto supone un
muy buen punto de partida para la araña y ya mejoraría a \textit{Tropescraper},
el cual tenía como punto de partida la entrada de la wiki de \textit{tropos} y
\textit{películas}, que no contiene todas las entradas existentes y son páginas
más difíciles de explorar al tener que entrar en sub índices según género o tipo
de media.

\subsection{Estructura de las páginas de TvTropes}
Por último, se analiza la estructura que tiene cada página de TvTropes, tanto de
un \textit{tropo} como de una obra audiovisual, ya sea película, serie,
videojuego, etc. Se intentan definir los tipos de páginas más comunes e
importantes, pero no todos los que puedan existir. Es imposible predecir todas
las formas que puede tomar una página, y además eso haría que el
\textit{scraper} no supiese adaptarse y fuese demasiado rígido
\cite{nishalscraping}. En general, en esta última sección se busca identificar
las partes más destacables del código HTML de las páginas para que el
\textit{scraper} sepa dónde tiene que buscar la información que necesita sin
redundancia.

Las páginas de obras audiovisuales, independientemente del tipo, se tratan
esencialmente igual y podrían presentar las mismas diferencias que dos del mismo
tipo, sin embargo, suelen tener secciones adicionales. Por ejemplo, si la página
es de una película, es probable que contenga la lista de actores, cosa que por
ejemplo no pasará en un libro o videojuego generalmente, no obstante, en general
entre los cambios que se pueden observar al buscar dos páginas que hacen
referencia a un mismo tema como puede ser dos obras audiovisuales el propósito
es el mismo: describir la obra y sus \textit{tropos}.

Por ejemplo, la página de la serie Los
Soprano\footnote{\url{https://tvtropes.org/pmwiki/pmwiki.php/Series/TheSopranos}}
presenta varias características interesantes, como por ejemplo que el
\textit{tropo}
\begin{otherlanguage}{english}\textit{GenreDeconstruction}\end{otherlanguage}\footnote{\url{https://tvtropes.org/pmwiki/pmwiki.php/Main/GenreDeconstruction}}
no está presente en la sección de \textit{tropos}, que está organizada en cuatro
carpetas según el rango alfabético, sino que se encuentra referenciado en la
propia descripción de la serie. También se pueden encontrar otros
\textit{tropos} que, si bien no están en la propia lista, se les hace referencia
en la descripción de un \textit{tropo} concreto que sí que está en la lista. Al
entrar en otra página, como la de la película El Viaje de
Chihiro\footnote{\url{https://tvtropes.org/pmwiki/pmwiki.php/Anime/SpiritedAway}},
podemos ver que la información ha cambiado por completo de forma y ahora los
\textit{tropos} no se presentan en carpetas y no están organizados por género o
cualquier otra idea, simplemente están en una lista ordenada alfabéticamente.
Por último, es interesante observar una página más, la de la serie
Hannibal\footnote{\url{https://tvtropes.org/pmwiki/pmwiki.php/Series/Hannibal}},
que aunque inicialmente puede parecer que tiene la misma estructura que la de
Los Soprano con cuatro carpetas por rango alfabético, observamos que esos rangos
son distintos. Esta última serie también evidencia otro problema de estructura
que es necesario resolver, y es que existen otras entradas en TvTropes que
también se llaman Hannibal, haciendo referencia o al personaje o a la novela, y
que tienen distinta manera de presentar sus tropos. Sin embargo, cada una de
estas páginas tienen un tipo definido en la web (personaje, serie, libro,
película, etc.) por lo que son más fácilmente identificables.

Como se ha explicado antes, estos cambios estructurales y de organización están
presentes también entre páginas con distintos propósitos, como por ejemplo, en
la página de un \textit{tropo} y de una obra. La página de una obra concreta
muestra los \textit{tropos} organizados en carpetas o listas con distintas
organizaciones, mientras que una página de un \textit{tropo} concreto como el
visto anteriormente de
\begin{otherlanguage}{english}\textit{GenreDeconstruction}\end{otherlanguage} da
la información inversa, es decir, todas las obras que tienen ese \textit{tropo}.
Sin embargo, en este caso la información está presentada de una forma
completamente distinta, esta vez requiere de una mayor exploración a fondo,
puesto que, se dan una serie de géneros con un hipervínculo a sub páginas que
cuelgan de la principal del \textit{tropo} y que ya sí que listan las obras.

En general, los \textit{tropos} pueden aparecer en casi cualquier parte dentro
de la página de una obra y estar organizados de distintas maneras. Algunas los
tienen organizados en una lista o por carpetas alfabéticas, por géneros o
cualquier otro método de diferenciarlos. También puede haber información de
\textit{tropos} anidada dentro de otros, referenciados en la descripción de la
obra, o incluidos en sub páginas de distinta índole.

En cuanto a las sub páginas dentro de una obra, estas pueden ser muchas o pocas
según la obra, ya que, también las crea la comunidad. Cualquiera de estas sub
páginas busca dar información interesante relacionada con una obra, pero
centrado en distintas temáticas o con distinto objetivo y, por tanto, presentan
información que puede ser novedosa. Existen varias muy comunes, como
\textit{Laconic}, que da una descripción muy breve de la obra referenciando
varios \textit{tropos} que generalmente no suelen aparecer en la lista
principal, \textit{TearJerker}, que referencia mediante \textit{tropos} aspectos
tristes de la obra, o \textit{Trivia}, que presenta curiosidades sobra la obra y
\textit{tropos} que tienen que ver con esas curiosidades. Otra de las sub
páginas más destacables es la llamada
YMMV\footnote{\url{https://tvtropes.org/pmwiki/pmwiki.php/YMMV/HomePage}}, la
cual contiene otros \textit{tropos} que no toda la comunidad identifica como
correctos o con el mismo significado y, por tanto, no corresponden en la página
principal, que lista los que no tienen diferencias de opinión. Definir todas las
sub páginas de cada obra es imposible, por lo que el \textit{scraper} debe saber
identificar la sección en la que aparecen todas estas páginas, explorarlas y
buscar enlaces de \textit{tropos}.

Identificar cuándo se hace referencia a un \textit{tropo}, independientemente de
la página, es sencillo puesto que siempre son hipervínculos con la clase CSS
``\textit{twikilink}'' que, además, siguen el mismo estilo de URL que se ha
visto anteriormente. El \textit{scraper} debe buscar, siempre dentro de la misma
página de la obra, todas las secciones y sub páginas que puedan contener un
hipervínculo de tipo ``\textit{twikilink}'', extraer el nombre del
\textit{tropo} y añadirlo sin repetir.

Finalmente, se resumen los puntos principales que se deducen de este análisis de
TvTropes:
\begin{itemize}
    \item Se considera que todos los \textit{tropos} de una obra audiovisual son
    la suma de todos los que aparecen listados en la página principal, los que
    aparecen referenciados en las distintas sub páginas dentro de la propia obra
    y los que aparecen referenciados en cualquier texto, ya sea el resumen de la
    obra o la descripción de otro \textit{tropo}.
    \item Un \textit{tropo} se identifica al ser una etiqueta HTML \texttt{<a
    class=``twikilink''>}. Todos los \textit{tropos} tienen una URL de la forma
    \texttt{https://tvtropes.org/pmwiki/pmwiki.php/Main/} más su nombre en
    estilo \textit{CamelCase}, sin embargo, no todas las páginas con este tipo
    de URL son \textit{tropos}.
    \item Existen índices que contienen completamente todos los tropos y obras
    audiovisuales separadas por tipo, listados alfabéticamente y sin confusión
    de pertenecer a cualquier otra categoría. Por tanto, se sabe sin lugar a
    dudas si una página es un \textit{tropo} si está contenido en ese índice.
\end{itemize}