\chapter{Análisis del problema}
En este capítulo se modela el problema de construcción de un \textit{scraper} de
TvTropes aplicando el patrón de \textit{Domain Driven Design}. Luego, se analiza
la estructura que sigue la web, identificando los puntos más importantes para
saber cómo abordar la creación de la araña que explore todas las páginas
relevantes de TvTropes y la extracción de la información que queremos en cada
una de ellas. El objetivo general del capítulo es servir de inicio para saber
cómo abordar la solución al problema y así poder llevarlo luego a lo práctico en
la implementación del \textit{software} en el capítulo 6, en el que se
desarrollará la estrategia final para resolver el problema de la extracción de
datos en TvTropes.

\section{Modelado del problema}

\textit{Domain Driven Design}

\section{Análisis de TvTropes}
TvTropes es un wiki complejo en el que tanto la estructura concreta de una
página como entre ellas cambia bastante; la información está desperdigada por
distintos lugares y no siempre toda la información que necesitamos es fácilmente
accesible o existe siquiera. A continuación, tras haber visto cómo se pretende
modelar el problema y organizar el código, se estudian los problemas y
particularidades que tiene TvTropes y que queremos resolver con el
\textit{scraper} para poder extraer toda la información que sea posible de la
manera más sencilla y eficiente. Para ello, se estudia la organización de las
páginas de TvTropes tanto entre sí como dentro de ellas, y se observan los
puntos más importantes que el bot debe saber identificar, para poder desarrollar
una estrategia en el siguiente capítulo conociendo todas las características de
la web.

Primero, se hará una breve introducción sobre cómo funciona TvTropes en general
y a nivel de usuario. Luego, se analizará la organización entre páginas para el
desarrollo del \textit{crawler} o araña. Finalmente, se analizará la estructura
del código HTML dentro de una misma página, para el desarrollo del
\textit{scraper}.

\subsection{Organización general de TvTropes}
Antes de pasar a entender en profundidad cómo estan relacionadas las partes de
TvTropes, es necesario primero ver cómo funciona la web en general y qué
contenidos ofrece.

En este trabajo el interés está en identificar y extraer los \textit{tropos},
que son la principal fuente de información de TvTropes.

Al ser la principal información de la web, todos los \textit{tropos} están .
Este índice se verá más en profundidad en la siguiente subsección, pero por lo
pronto se puede ver que todos siguen en su URL la misma forma que consiste en
\texttt{https://tvtropes.org/pmwiki/pmwiki.php/Main/} más el nombre del tropo en
estilo \textit{CamelCase}. Es decir, todos los \textit{tropos} cuelgan de
\textit{Main} y están escritos de la forma \texttt{NombreDelTropo}.

\subsection{El indexado en TvTropes}
Para construir una buena araña que sea capaz de explorar todas las páginas de
TvTropes es necesario entender cómo están relacionadas entre sí las páginas.
Para esto es necesario conocer cómo funciona el indexado en TvTropes, es decir,
de qué manera están organizadas todas las secciones de la web y cómo un usuario
puede encontrar lo que quiere. Al saber cómo un usuario puede encontrar lo que
quiere, también sabemos cómo debe comportarse el \textit{crawler} para indexar
las páginas que necesitamos.

La araña necesita un buen punto de partida a partir del cual, indexando todos
los hipervínculos que encuentre, pueda ir explorando recursivamente y encontrar
todas las páginas que necesita luego el \textit{scraper}.

Como se ha visto antes, el índice más importante es el índice principal, o
\textit{Main}. De este índice cuelgan todos los \textit{tropos} y su URL tiene
siempre las mismas partes, solo cambiando el nombre del tropo. Sin embargo, para
encontrar los propios \textit{tropos} desde la web la organización es distinta.

Existe dos índices de \textit{tropos} ordenados alfabéticamente. Uno de esos
índices\footnote{\url{https://tvtropes.org/pmwiki/index_namespaces.php}} da una
barra de búsqueda en la que poder escribir el nombre del tropo que se busca y
una lista que, sin embargo, no contiene ni mucho menos todos los
\textit{tropos}. Al buscar, por ejemplo, aquellos que empiecen por la letra Z
vemos que sólo se listan tres, mientras que con una breve búsqueda en cualquier
obra se puede ver que hay muchísimos otros con la letra Z que no están listados.
Esto evidencia que en este índice faltan muchos \textit{tropos} y no puede
usarse como un buen punto de partida para ellos. Por suerte, existe otro índice
que sí que los contiene
todos\footnote{\url{https://tvtropes.org/pmwiki/pagelist_having_pagetype_in_namespace.php?n=Main&t=trope}}
ya que es un script interno de TvTropes que hace una búsqueda en toda la web con
páginas de un tipo llamado \textit{trope}. Por tanto, se debería tener en cuenta
este último ya que nos asegura que están listados todos los existentes en la
página y además es el que se recomienda en los
foros\footnote{\url{https://tvtropes.org/pmwiki/posts.php?discussion=14420393930A40584600&page=1}}.

Por último, se puede observar que la URL de este último índice contiene
parámetros que, si se modifican, nos permiten obtener un índice completo de, por
ejemplo, todas las
películas\footnote{\url{https://tvtropes.org/pmwiki/pagelist_having_pagetype_in_namespace.php?n=Film&t=work}}
o todos los
videojuegos\footnote{\url{https://tvtropes.org/pmwiki/pagelist_having_pagetype_in_namespace.php?n=Videogame&t=work}}.

En resumen, se han podido encontrar índices que sabemos con toda seguridad que
contienen todos los tropos y obras separadas por tipo, por lo que esto supone un
muy buen punto de partida para la araña y ya mejoraría a \textit{Tropescraper},
el cual usaba como punto de partida la entrada de la wiki de \textit{tropos} y
\textit{películas}, que no contiene todas las entradas existentes y son páginas
más difíciles de explorar al tener que entrar en sub índices según género o tipo
de media.

\subsection{Estructura de las páginas de TvTropes}
Por último, se analiza la estructura que tiene cada página de TvTropes, tanto de
un \textit{tropo} como de una obra audiovisual, ya sea película, serie,
videojuego, etc. Se intentan definir los tipos de páginas más comunes e
importantes, pero no todos los que puedan existir. Es imposible predecir todas
las formas que puede tomar una página, y además eso haría que el
\textit{scraper} no supiese adaptarse y fuese demasiado rígido
\cite{nishalscraping}. En general, en esta última sección se busca identificar
las partes más importantes del código HTML de las páginas para que el
\textit{scraper} sepa dónde tiene que buscar la información que necesita sin
redundancia.

Las páginas de obras audiovisuales, independientemente del tipo, se tratan de
forma esencialmente igual y podrían presentar las mismas diferencias que dos del
mismo tipo, sin embargo, suelen tener secciones adicionales. Por ejemplo, si la
página es de una película, es probable que contenga la lista de actores, cosa
que por ejemplo no pasará en un libro o videojuego generalmente, pero en general
entre los cambios que se pueden observar al buscar dos páginas que hacen
referencia a un mismo tema como puede ser dos obras audiovisuales el propósito
es el mismo: describir la obra y sus \textit{tropos}.

Por ejemplo, la página de la serie Los
Soprano\footnote{\url{https://tvtropes.org/pmwiki/pmwiki.php/Series/TheSopranos}}
presenta varias características interesantes, como por ejemplo que el
\textit{tropo}
\begin{otherlanguage}{english}\textit{GenreDeconstruction}\end{otherlanguage}\footnote{\url{https://tvtropes.org/pmwiki/pmwiki.php/Main/GenreDeconstruction}}
no está presente en la sección de \textit{tropos}, que está organizada en cuatro
carpetas según el rango alfabético, sino que se encuentra referenciado en la
propia descripción de la serie. También se pueden encontrar otros
\textit{tropos} que no están en la propia lista, pero a los que se hace
referencia en la descripción de un \textit{tropo} concreto que sí que está en la
lista. Al entrar en otra página, como la de la película El Viaje de
Chihiro\footnote{\url{https://tvtropes.org/pmwiki/pmwiki.php/Anime/SpiritedAway}},
podemos ver que la información ha cambiado por completo de forma y ahora los
\textit{tropos} no se presentan en carpetas y no están organizados por género o
cualquier otra idea, simplemente están en una lista ordenada alfabéticamente.
Por último, es interesante observar una página más, la de la serie
Hannibal\footnote{\url{https://tvtropes.org/pmwiki/pmwiki.php/Series/Hannibal}},
que aunque inicialmente puede parecer que tiene la misma estructura que la de
Los Soprano con cuatro carpetas por rango alfabético, observamos que esos rangos
son distintos. Esta última serie también evidencia otro problema de estructura
que es necesario resolver, y es que existen otras entradas en TvTropes que
también se llaman Hannibal, haciendo referencia o al personaje o a la novela, y
que tienen distinta manera de presentar sus tropos. Sin embargo, cada una de
estas páginas tienen un tipo definido en la web (personaje, serie, libro,
película, etc.) por lo que son más fácilmente identificables.

Como se ha explicado antes, estos cambios estructurales y de organización están
presentes también entre páginas con distintos propósitos, como por ejemplo, en
la página de un \textit{tropo} y de una obra. La página de una obra concreta
muestra los \textit{tropos} organizados en carpetas o listas con distintas
organizaciones, mientras que una página de un \textit{tropo} concreto como el
visto anteriormente de
\begin{otherlanguage}{english}\textit{GenreDeconstruction}\end{otherlanguage} da
la información inversa, es decir, todas las obras que tienen ese \textit{tropo}.
Sin embargo, en este caso la información está presentada de una forma
completamente distinta, esta vez requiere de una mayor exploración a fondo,
puesto que, se dan una serie de géneros con un hipervínculo a sub páginas que
cuelgan de la principal del \textit{tropo} y que ya sí que listan las obras.

Algunas teniendo los \textit{tropos} organizados por carpetas alfabéticas o por
géneros, otras teniéndolos en una lista; información de tropos anidada dentro de
otros o contenida en sub páginas llamadas
YMMV\footnote{\url{https://tvtropes.org/pmwiki/pmwiki.php/YMMV/HomePage}}, las
cuales contienen otros \textit{tropos} que no toda la comunidad identifica como
correctos o con el mismo significado y, por tanto, no corresponden en la página
principal, que lista los que no tienen diferencias de opinión. Estos son solo
varios de los muchos otros métodos de organización que requieren de un análisis
para poder adaptarse a las plantillas que presenta la web para dar su
información y el \textit{scraper} pueda entenderlos.

Identificar cuándo se hace referencia a un \textit{tropo}, independientemente de
la página, es sencillo puesto que siempre son hipervínculos con la clase CSS
``\textit{twikilink}'' que, además, siguen el mismo estilo de URL que se ha
visto anteriormente.

Finalmente, se resumen los puntos principales que se deducen de este análisis de
TvTropes:
\begin{itemize}
    \item Se considera que todos los \textit{tropos} de una obra audiovisual son
    la suma de todos los que aparecen listados en la página principal y los que
    aparecen referenciados en las distintas sub páginas dentro de la propia
    obra.
\end{itemize}