\chapter{Análisis del problema}
En este capítulo se modela el problema de construcción de un \textit{scraper} de
TvTropes aplicando el patrón de \textit{Domain Driven Design}. Luego, se analiza
la estructura que sigue la web, desarrollando una estrategia con la que saber
cómo abordar la creación de la araña que explore todas las páginas relevantes de
TvTropes y la extracción de la información que queremos en cada una de ellas. El
objetivo del capítulo es servir de preludio al capítulo 6, en el que se
implementará finalmente el \textit{scraper} de TvTropes.

\section{Modelado del problema}

\textit{Domain Driven Design}

\section{Análisis de la estructura de TvTropes}
TvTropes es un wiki complejo en el que la estructura de una página y entre ellas
cambia bastante, la información está desperdigada por distintos lugares y no
siempre toda la información que necesitamos es fácilmente accesible o existe
siquiera. A continuación, tras haber visto cómo se pretende modelar el problema
y organizar el código, se estudian los problemas y particularidades que tiene
TvTropes y que queremos resolver con el \textit{scraper} para poder extraer toda
la información que sea posible de la manera más sencilla y eficiente. Para ello,
se estudia la organización de las páginas de TvTropes tanto entre sí como dentro
de ellas, y se observan los puntos más importantes que el \textit{scraper} debe
saber identificar, a modo de estrategia para poder desarrollarlo en el siguiente
capítulo.

\subsection{El índice de TvTropes}

\subsection{Estructura de las páginas de TvTropes}
Estos cambios se pueden observar al buscar dos páginas que hacen referencia a un
mismo tema como puede ser dos obras audiovisuales, no importa si son del mismo
medio o no, el propósito es el mismo: describir la obra y sus \textit{tropos}.
Por ejemplo, la página de la serie Los
Soprano\footnote{\url{https://tvtropes.org/pmwiki/pmwiki.php/Series/TheSopranos}}
presenta varias características interesantes, como por ejemplo que el
\textit{tropo}
\begin{otherlanguage}{english}\textit{GenreDeconstruction}\end{otherlanguage}\footnote{\url{https://tvtropes.org/pmwiki/pmwiki.php/Main/GenreDeconstruction}}
no está presente en la sección de \textit{tropos}, que está organizada en cuatro
carpetas según el rango alfabético, sino que se encuentra referenciado en la
propia descripción de la serie. También se pueden encontrar otros
\textit{tropos} que no están en la propia lista, pero a los que se hace
referencia en la descripción de un \textit{tropo} concreto que sí que está en la
lista. Al entrar en otra página, como la de la película El Viaje de
Chihiro\footnote{\url{https://tvtropes.org/pmwiki/pmwiki.php/Anime/SpiritedAway}},
podemos ver que la información ha cambiado por completo de forma y ahora los
\textit{tropos} no se presentan en carpetas y no están organizados por género o
cualquier otra idea, simplemente están en una lista ordenada alfabéticamente.
Por último, es interesante observar una página más, la de la serie
Hannibal\footnote{\url{https://tvtropes.org/pmwiki/pmwiki.php/Series/Hannibal}},
que aunque inicialmente puede parecer que tiene la misma estructura que la de
Los Soprano con cuatro carpetas por rango alfabético, observamos que esos rangos
son distintos. Esta última serie también evidencia otro problema de estructura
que es necesario resolver, y es que existen otras entradas en TvTropes que
también se llaman Hannibal, haciendo referencia o al personaje o a la novela, y
que tienen distinta manera de presentar sus tropos. Sin embargo, cada una de
estas páginas tienen un tipo definido en la web (personaje, serie, libro,
película, etc.) por lo que son más fácilmente identificables.

Como se ha explicado antes, estos cambios estructurales y de organización están
presentes también entre páginas con distintos propósitos, como por ejemplo, en
la página de un \textit{tropo} y de una obra. La página de una obra concreta
muestra los \textit{tropos} organizados en carpetas o listas con distintas
organizaciones, mientras que una página de un \textit{tropo} concreto como el
visto anteriormente de
\begin{otherlanguage}{english}\textit{GenreDeconstruction}\end{otherlanguage} da
la información inversa, es decir, todas las obras que tienen ese \textit{tropo}.
Sin embargo, en este caso la información está presentada de una forma
completamente distinta, esta vez requiere de una mayor exploración a fondo,
puesto que, se dan una serie de géneros con un hipervínculo a sub páginas que
cuelgan de la principal del \textit{tropo} y que ya sí que listan las obras.

Algunas teniendo los \textit{tropos} organizados por carpetas alfabéticas o por
géneros, otras teniéndolos en una lista; información de tropos anidada dentro de
otros o contenida en sub páginas llamadas
YMMV\footnote{\url{https://tvtropes.org/pmwiki/pmwiki.php/YMMV/HomePage}}, las
cuales contienen otros \textit{tropos} que no toda la comunidad identifica como
correctos o con el mismo significado y, por tanto, no corresponden en la página
principal, que lista los que no tienen diferencias de opinión. Estos son solo
varios de los muchos otros métodos de organización que requieren de un análisis
para poder adaptarse a las plantillas que presenta la web para dar su
información y el \textit{scraper} pueda entenderlos.

Finalmente, se resumen los puntos principales que se deducen de este análisis de
TvTropes:
\begin{itemize}
    \item 
\end{itemize}