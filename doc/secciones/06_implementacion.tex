\chapter{Implementación}
\label{chapter:6}

Este capítulo relata el proceso completo de desarrollo del \textit{software
scraper} para TvTropes que resuelve los problemas y cumple los objetivos
definidos en el \autoref{chapter:2}. Se siguen uno a uno los distintos hitos que
se han ido alcanzando, explicando por qué son necesarios y las decisiones que se
han tomado en cada uno de ellos para cumplir las historias de usuario, que guían
todo el desarrollo con agilidad.

En este capítulo se llevan a la práctica todas las buenas prácticas,
recomendaciones y herramientas justificadas en el \autoref{chapter:4} para
llevar un desarrollo ágil. Esto implica, además, que todos los hitos se
desarrollan con el principal objetivo de satisfacer al usuario y asegurar la
calidad, por lo que conforme se justifiquen las decisiones tomadas para avanzar
en cada uno de ellos, se relacionarán con las historias de usuario definidas,
que son las que confirman qué se debe implementar para cumplir con las
expectativas del usuario.

\section{Comienzo del desarrollo}
Con el primer modelado del problema que se ha realizado en el
\autoref{chapter:5}, tenemos definido todo lo necesario para poder comenzar con
el desarrollo y dar una primera versión de un \textit{scraper} que sea capaz de
extraer la información de \textit{tropos} contenida en páginas individuales de
películas de TvTropes. El primer \textit{milestone} de desarrollo es el
\href{https://github.com/jlgallego99/TropesToGo/milestone/3}{M2}, que resuelve
la parte previa a la extracción. Antes de que el \textit{scraper} extraiga la
información que necesita el usuario, que es el objetivo del siguiente hito,
tiene que determinar si la página sobre la que está trabajando es extraíble, es
decir, pertenece a una página de TvTropes que describe una obra con sus
\textit{tropos} y tiene una estructura HTML conocida. Esto permite determinar,
antes de llevar a cabo el proceso de extracción, que el wiki de TvTropes no ha
cambiado demasiado, lo suficiente para saber si el \textit{scraper} programado
sigue funcionando y así evitar procesamiento innecesario. 

Puesto que se está construyendo un producto mínimo, este hito se centra en que
las páginas de obras que se analicen sean solamente de películas, teniendo un
\textit{milestone} futuro dedicado a ampliarlo a otros tipos de medios
audiovisuales. Solo es necesario comprobar y extraer páginas de un tipo de medio
audiovisual concreto para determinar que el \textit{scraper} completa
correctamente sus tareas, ya que el resto de medios consisten en explorar otras
páginas con una estructura similar y que se obtendrán una vez se tenga una araña
que encuentre ese tipo de páginas. Se han elegido páginas de películas por ser
el medio en el que se centran la mayoría de trabajos de investigación estudiados
en la introducción y el estado del arte, y son las que más interés tienen para
una primera versión funcional.

\subsection{Flujo de integración continua para pruebas y compilación}
En este hito se empieza a programar código funcional con lógica de negocio, así
que se comienzan también a desarrollar conjuntamente los tests, concretamente
antes de la propia funcionalidad, tal y como se especifica en el desarrollo
dirigido por pruebas \cite{beck2002driven} y se definió en el
\autoref{chapter:4}. Por tanto, uno de los objetivos de este hito es tener
integrado en el código del proyecto el \textit{framework} de pruebas Ginkgo, que
es el que permitirá desarrollar los tests en Go, para que en los próximos hitos
se puedan desarrollar todas las pruebas para testear las nuevas funcionalidades.
Además, al estar en un entorno de desarrollo ágil la ejecución de las pruebas
estará automatizada, de forma que se tenga un flujo de integración continua que
ejecute automáticamente todos los tests definidos y se pueda comprobar desde el
repositorio de GitHub. 

En cada uno de los hitos, cuando se desarrolla nueva funcionalidad, se abre un
\textit{pull request} en GitHub con los cambios realizados en el proyecto para
saber qué incremento en el producto se está haciendo y qué \textit{issues}
pretende resolver para avanzar en completar el hito. Los nuevos flujos de CI que
se añaden en este hito permiten automatizar la ejecución de las pruebas y
compilar todo el proyecto, para cumplir que todo el código sea funcional y
asegurar su calidad. Si cualquiera de estos flujos de CI fallan, no se podrá
aceptar el \textit{pull request} y añadir los cambios a la rama principal. La
integración continua en este proyecto se configura mediante \textit{GitHub
Actions}, que automatizan los procesos apoyándose en nuevas tareas definidas
mediante el gestor de tareas \textit{Mask}. Estas tareas ejecutan los tests para
todo el proyecto, ambas usando por debajo la propia utilidad del lenguaje Go:
\texttt{go test} para las pruebas y \texttt{go build} para comprobar que todos
los paquetes del código compilan correctamente. Esto permite que no haga falta
modificarlos en el futuro; servirán para todo el desarrollo de ahora en adelante
y ejecutarán automáticamente todas las nuevas pruebas que se programen. En el
caso de necesitar algún cambio en la ejecución de los tests o la compilación,
bastará con cambiar las tareas del gestor de tareas, sin necesidad de tocar la
configuración de la integración continua.

Por tanto, para poder cumplir con la automatización de las pruebas y la
compilación se configura un nuevo \textit{GitHub Action} en el que se prepara
Go, se instala el gestor de tareas \textit{Mask} y se ejecutan las tareas
definidas para la compilación y el testeo. En un principio se probó a instalar
\textit{Mask} mediante \textit{cargo}, el gestor de paquetes del lenguaje Rust,
ya que la máquina virtual que utiliza el \textit{Action} es de Ubuntu y su
gestor de paquetes no tiene disponible este gestor de tareas en su repositorio
para instalarlo. Sin embargo, esta manera de instalar el gestor de tareas
suponía una gran sobrecarga en el flujo de trabajo al tener que instalar todo el
lenguaje Rust únicamente para instalar un paquete, tardando casi 2 minutos en
ejecutarlo por completo, gastando la mayoría de recursos en instalar el gestor
de tareas. Para solucionar esto se acabó optando por introducir los comandos en
el flujo de CI necesarios para descargar directamente el binario de
\textit{Mask}; esto hizo que se mejorase el tiempo en más de la mitad, ahorrando
los recursos de GitHub y teniendo el resultado del flujo de CI mucho antes.
Adicionalmente, una de las ventajas que tiene el usar Go es que, al ejecutar
cualquier comando de compilación, testeo o ejecución del código, comprueba
automáticamente si las dependencias definidas en el fichero \texttt{go.mod}
están instaladas, y si no lo están las descarga automáticamente, por lo que no
hace falta especificar explícitamente qué dependencias hay que instalar para que
el código funcione.

\subsection{Comprobación de la estructura de una página de TvTropes}
En este hito se modifica el servicio \textit{scraper} añadiendo la funcionalidad
necesaria para cumplir su objetivo: verificar que una página de una obra de
TvTropes no ha cambiado y se puede extraer. Un servicio en DDD hace referencia a
un paquete de código que ejecuta cierta lógica de negocio para un cliente y en
este caso el cliente sería el usuario que necesita los datos. 

El objetivo de este hito se alcanza implementando la función \texttt{CheckValidWorkPage} que encapsula la funcionalidad de revisar toda la
estructura de una página de película de TvTropes y validar si el
\textit{scraper} podrá extraerla. Recibe una entidad \textit{Page}, de la cual
el \textit{scraper} extrae lo principal que necesita, que es el URL de la página
para poder hacerle una petición y obtener su código HTML. La razón de recibir la
página por referencia se debe a que la página es mutable, es posible que se haya
actualizado y el \textit{scraper} necesite en el futuro cambiar su atributo de
última modificación, además de que será más eficiente para el \textit{scraper}
el obtener las páginas por referencia que previamente habrá creado el
\textit{crawler} en hitos posteriores. En general, en DDD se tratan las
entidades siempre como referencias, puesto que son mutables, mientras que los
objetos valor se tratarán como variables instanciadas. 

Esta función se ha modularizado en distintas subfunciones que realizan
revisiones de distintos tipos, cumpliendo la propiedad de única
responsabilidad de cada una de las funciones y haciéndolas más legibles. Esto
facilita que, en caso de que se necesitasen analizar nuevas partes, se modifiquen
las sub funciones ya existentes o se añada una nueva, que simplemente se
llamaría desde el método principal. Según \textit{clean code} no se deben tener
funciones que mezclen llamadas a otras funciones de alto nivel con lógica más
compleja, por lo que el método principal \texttt{CheckValidWorkPage} se encarga
de llamar a las distintas sub funciones que llevan a cabo verificaciones de distintos
tipos y trata sus resultados para determinar finalmente si la información de la
página es extraíble o no. Estas funciones son:
\begin{itemize}
    \item La función \texttt{CheckTvTropesWorkPage} analiza el URL de la página
    para comprobar si el \textit{host} es \url{tvtropes.org} y si el URI
    corresponde a lo que entendemos como una página de película, es decir,
    pertenece a \textit{Main} y está bajo el índice \textit{Film}. La penúltima
    parte de la ruta tiene que ser el índice al que pertenece la obra, en este
    caso \textit{Film}, y la última parte será el nombre de la película. Esto es
    lo primero que se debe hacer; si la página que se está analizando no es de
    TvTropes y/o no pertenece a una obra, no tiene sentido seguir adelante con
    ella.
    \item La función \texttt{CheckMainArticle} itera por el árbol DOM construido
    a partir del HTML de la página y va viendo si sigue la estructura de una
    página de una obra en el wiki de TvTropes. Principalmente, revisa si la
    página tiene las partes vistas en el análisis de la Figura
    \ref{fig:tvtropes-work}: un artículo principal con sus identificadores y
    clases conocidas y si el índice, dentro del título del artículo, pertenece
    al de películas.
    \item Por último, se tiene la función \texttt{CheckTropeSection} que realiza
    verificaciones más complejas relacionadas con la principal sección que se
    querrá analizar: la sección de \textit{tropos}. Analiza las tres formas más
    comunes que toma la sección de \textit{tropos} y que se pueden ver en la
    Figura \ref{fig:tropelist} del \autoref{chapter:5}. En este primer hito solo
    se tienen en cuenta los tres tipos, ya que bastan para conformar un
    producto mínimo que sea capaz de analizar la mayoría de páginas de TvTropes,
    sin embargo, en futuros hitos se tendrán en cuenta casos más extremos.\\
    La función primero identifica la sección donde se encuentran los
    \textit{tropos}, que es una lista sin ordenar dentro del artículo y que está
    siempre después del resumen y precedida de una etiqueta de \textit{header}.
    Una vez confirmado que la página tiene esa sección, se ocupa de ver si se
    adapta a uno de los tres tipos que se conocen. Identifica si los
    \textit{tropos} están contenidos en carpetas buscando el botón de abrir
    carpetas, que sabemos su etiqueta, clase y función JavaScript que utiliza
    para abrirlas, al estar generado siempre igual por el motor del wiki. En
    todos los casos se comprueba si el primer elemento de la lista es un enlace
    a una página de \textit{tropos} o a una sub página de ellos. Para este
    último caso, que pertenecería al tercer tipo, se ha hecho uso de una
    expresión regular que confirma si el URI al que redirigen sigue un patrón
    concreto. Si la última parte del URI está codificada como la cadena
    \texttt{/<NombreDeLaObra>/Tropes<XtoY>} entonces se acepta como algo
    conocido. Si la lista contiene cualquier otro elemento inesperado, se
    considera que la página tiene una estructura desconocida y no se puede
    extraer su información.
\end{itemize}

Todas estas sub funciones no forman parte de la interfaz de funciones del servicio
ni están exportadas al ser simplemente funciones internas solo para el
\textit{scraper}, por lo que no se pueden llamar fuera de su paquete. Además,
todas ellas devuelven un valor booleano, que indica si la página es extraíble o
no, y un error que indica el por qué no es extraíble en caso de no serlo. En
general, es idiomático en Go que todas las funciones devuelvan siempre un error
como último valor para que el código sea autoexplicativo y se tenga un completo
y correcto control de los errores tanto para que las funciones de testing puedan
comprobar todos los casos como para que cualquier cliente que llame a la función
del servicio entienda exactamente qué ha pasado en caso de error
\cite{effective_go}.

Finalmente, en este hito se toma la decisión de que los selectores CSS, los
cuales son una cadena de \textit{string} que hacen referencia a varias etiquetas
HTML y clases, se encapsularán en valores constantes. Esto sigue las guías de
\textit{clean code} sobre evitar los literales, facilitando la legibilidad del
código y evitando la repetición. De este modo, el \textit{scraper} pasará como
parámetro estas constantes a las funciones de Goquery que encuentran los nodos
del árbol DOM de la página que se está explorando. A estas constantes se les da
un nombre lo más semántico posible de forma que el código sea legible y limpio.
Por ejemplo, la constante \texttt{TropeTag} contiene el selector \texttt{a.twikilink} que hace referencia a todos los \textit{anchor} de HTML
cuya clase es \textit{twikilink}, que sabemos debido al análisis del capítulo
anterior que hacen referencia a enlaces hacia páginas de \textit{tropos}. El
tener constantes con selectores importantes para lo que quiere buscar el
\textit{scraper} también implica otras ventajas como el poder combinar distintos
selectores de modo que se entienda su significado completo con solo leerlo o el
poder adaptarse a los futuros y posibles cambios que pueda sufrir TvTropes. En
el caso de que cambie alguna clase o etiqueta con la que encontrar alguna
sección valiosa de una página, baste con cambiar la constante y no cambiarlo
en cada parte del código en la que aparezca. En general, al usar constantes
semánticas para los selectores se facilita mucho la legibilidad de algo que de
otro modo requeriría de un análisis más exhaustivo para entender qué significa
el selector utilizado, ya que pueden volverse muy complejos al unir varias
reglas.

\section{Extracción de información en la página de una película}

\section{Desarrollo de la araña}

\subsection{Estrategia general para la extracción de información de TvTropes}
El crawler primero indexa todas las páginas relevantes, el scraper luego extrae
la información y almacena en un fichero, etc.

\subsection{Arquitectura del \textit{crawler}}

\section{Aplicación de línea de comandos}