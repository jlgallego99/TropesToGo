\chapter{Implementación}
\label{chapter:6}

Este capítulo relata el proceso completo de desarrollo del \textit{software
scraper} para TvTropes que resuelve los problemas y cumple los objetivos
definidos en el \ref{chapter:2}. Se siguen uno a uno los distintos hitos que se
han ido alcanzando, explicando por qué son necesarios y las decisiones que se
han tomado en cada uno de ellos para cumplir las historias de usuario, que guían
todo el desarrollo con agilidad.

\section{Comienzo del desarrollo y comprobación de la estructura de una página de TvTropes}
Con el primer modelado del problema que se ha realizado en el \ref{chapter:5},
tenemos definido todo lo necesario para poder comenzar con el desarrollo y dar
una primera versión de un \textit{scraper} que sea capaz de extraer la
información de \textit{tropos} contenida en páginas individuales de películas de
TvTropes. El primer \textit{milestone} de desarrollo es el
\href{https://github.com/jlgallego99/TropesToGo/milestone/3}{M2}, que resuelve
la parte previa a la extracción. Antes de que el \textit{scraper} extraiga la
información que necesita el usuario, que es el objetivo del siguiente hito,
tiene que determinar si la página sobre la que está trabajando es extraíble, es
decir, pertenece a una página de TvTropes que describe una obra con sus
\textit{tropos} y tiene una estructura HTML conocida. Esto permite determinar,
antes de llevar a cabo el proceso de extracción, que el wiki de TvTropes no ha
cambiado demasiado, lo suficiente para saber si el \textit{scraper} programado
sigue funcionando y así evitar procesamiento innecesario. 

Puesto que se está construyendo un producto mínimo, este hito se centra en que
las páginas de obras que se analicen sean solamente de películas, teniendo un
\textit{milestone} futuro dedicado a ampliarlo a otros tipos de medios
audiovisuales. Solo es necesario comprobar y extraer páginas de un tipo de medio
audiovisual concreto para determinar que el \textit{scraper} completa
correctamente sus tareas, ya que el resto de medios consisten en explorar otras
páginas con una estructura similar y que se obtendrán una vez se tenga una araña
que encuentre ese tipo de páginas. Se han elegido páginas de películas por ser
el medio en el que se centran la mayoría de trabajos de investigación estudiados
en la introducción y el estado del arte, y son las que más interés tienen para
una primera versión funcional.

\section{Extracción de información en la página de una película}

\section{Desarrollo de la araña}

\subsection{Estrategia general para la extracción de información de TvTropes}
El crawler primero indexa todas las páginas relevantes, el scraper luego extrae
la información y almacena en un fichero, etc.

\subsection{Arquitectura del \textit{crawler}}

\section{Aplicación de línea de comandos}