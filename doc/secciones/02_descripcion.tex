\chapter{Descripción del problema}

En este capítulo se desarrollará el problema que se ha introducido en la sección
anterior y que se quiere resolver, formulando los objetivos que se quieren
alcanzar al final del proyecto.

\section{Problema a resolver}
La importancia de los \textit{tropos} y su estudio en multitud de ámbitos con
métodos de ciencia de datos, como el análisis de \textit{tropos} más usados en
películas y cuáles son los más populares a lo largo del tiempo
\cite{garcia2020tropes}, o el análisis mediante algoritmos de combinaciones de
\textit{tropos} que permiten conocer qué nichos narrativos no han sido
explorados aún y pueden reportar un buen beneficio y críticas positivas
\cite{garcia2021simpsons}, abren la necesidad de tenerlos todos bien recogidos y
preparados para poder realizar un buen estudio de ellos con las herramientas
actuales que existen en ciencia de datos o inteligencia artificial. 

Es necesario poder extraer estos contenidos reiterativos completos o bajo algún
tipo de criterio, si es que se desea analizar un subconjunto de ellos. Y a esto
se le suma también el interés que pueden tener los metadatos de una obra, siendo
un problema el tener extraída por ejemplo una película con sus \textit{tropos}
asociados, pero que no se tenga más información de esta como puede ser el año de
publicación, el género, los actores, etc. Podrían existir análisis que
requiriesen de todos estos metadatos, por lo que necesitarían que esta
información extraída sea identificable con la de otras bases de datos, como
puede ser IMDB\footnote{\url{https://www.imdb.com/}} para el ejemplo de una
película.

La estructura de las páginas de TvTropes y sus contenidos están en constante
cambio, varias páginas están estructuradas de distintas maneras, algunas usando
un esquema más antiguo, otras uno más nuevo; algunas teniendo los
\textit{tropos} organizados por carpetas alfabéticas, otras teniéndolos en una
lista; información de tropos anidada dentro de otros, y muchos otros métodos de
organización que requieren de un extenso análisis para poder adaptarse a todas
las formas que presenta la web para dar su información. Para solucionar estos
problemas el \textit{scraper} debe conocer de antemano todas las estructuras que
puede tomar una página de TvTropes y comprobar que la página que está explorando
se adapta a alguna de las existentes y, por tanto, su información es extraíble.
Además, extraer tanta cantidad de información presenta un reto adicional en
términos de eficiencia en el tiempo de ejecución debido a todo el tiempo que
puede tardar una herramienta en recoger tantos contenidos.

El problema principal que busca resolver este trabajo es el de desarrollar una
herramienta eficiente y eficaz mediante la cual cualquier persona interesada en
el estudio de los \textit{tropos} pueda obtener todos los datos de TvTropes
estructurados y preparados para su análisis, pudiendo elegir qué información
concreta quiere sin necesidad de extraerla toda y en qué formato de datos la
quiere representada. Esta herramienta debe tener en cuenta la naturaleza
cambiante tanto de los datos como de la fuente de información de donde los
extrae, y debe poder presentarlos en un formato legible tanto por humanos como
por programas, para poder hacer cualquier tipo de estudio sobre ellos. Estos
datos deberán ser completos, correctos y estar siempre actualizados para evitar
que cualquier tipo de análisis esté desfasado.

A continuación se describen los objetivos que busca alcanzar este proyecto tras
haber resuelto los problemas descritos. Estos objetivos se usarán en la última
sección de este trabajo como medida para comprobar si el proyecto ha finalizado
correctamente.

\section{Objetivos}
Una vez descrito el problema que se quiere resolver y los retos que presenta, se
desglosan los objetivos específicos que tiene este trabajo.

\begin{itemize}
    \item \textbf{OBJ01} Diseñar y desarrollar un \textit{scraper} que opere de
    forma autónoma y sea capaz de entender y extraer la información de cualquier
    página de \textit{tropos} de TvTropes independientemente de su estructura.
    \item \textbf{OBJ02} Diseñar y desarrollar una araña que sea capaz de
    explorar todas las URL de páginas de \textit{tropos} que existen en TvTropes
    para poder lanzar el \textit{scraper} en ellas.
    \item \textbf{OBJ03} Diseñar y desarrollar una aplicación de línea de
    comandos mediante la cual el usuario pueda interaccionar con el
    \textit{scraper} eligiendo qué contenidos quiere extraer.
    \item \textbf{OBJ04} El \textit{scraper} entenderá la frecuencia con la que
    cambian los datos de TvTropes y actualizará la información extraída.
    \item \textbf{OBJ05} El \textit{scraper} desarrollado debe ser rápido y
    eficiente, adaptándose a la gran cantidad de datos que tiene que extraer.
    \item \textbf{OBJ06} El \textit{scraper} extraerá información de metadatos
    no estructurada que permita poder diferenciar entre obras audiovisuales y
    relacionarlas con fuentes de datos externas.
    \item \textbf{OBJ07} Los datos extraídos por el \textit{scraper} se
    almacenarán en distintos formatos de representación de datos usados en el
    ámbito de la ciencia de datos.
\end{itemize}

