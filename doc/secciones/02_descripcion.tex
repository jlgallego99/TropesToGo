\chapter{Descripción del problema}

En este capítulo se desarrollará el problema que se ha introducido en la sección
anterior y que se quiere resolver, formulando los objetivos que se quieren
alcanzar al final del proyecto.

\section{Problema a resolver}
La importancia de los \textit{tropos} y su estudio en multitud de ámbitos con
métodos de ciencia de datos, como el análisis de \textit{tropos} más usados en
películas y cuáles son los más populares a lo largo del tiempo
\cite{garcia2020tropes}, o el análisis mediante algoritmos de combinaciones de
\textit{tropos} que permiten conocer qué nichos narrativos no han sido
explorados aún y pueden reportar un buen beneficio y críticas positivas
\cite{garcia2021simpsons}, abren la necesidad de tenerlos todos bien recogidos y
preparados para poder realizar un buen estudio de ellos con las herramientas
actuales que existen en ciencia de datos o inteligencia artificial. 

Es necesario poder extraer estos contenidos reiterativos completos o bajo algún
tipo de criterio, si es que se desea analizar un subconjunto de ellos. Y a esto
se le suma también el interés que pueden tener los metadatos de una obra, siendo
un problema el tener extraída por ejemplo una película con sus \textit{tropos}
asociados, pero que no se tenga más información de esta como puede ser el año de
publicación, el género, los actores, etc. Podrían existir análisis que
requiriesen de todos estos metadatos, por lo que necesitarían que esta
información extraída sea identificable con la de otras bases de datos, como
puede ser IMDB\footnote{\url{https://www.imdb.com/}} para el ejemplo de una
película.

La relación entre páginas de TvTropes, la información que contienen y la
disposición de sus contenidos están en constante cambio. Estos cambios se pueden
observar en cualquier página de la web, ya que, no tiene una estructura fija,
sino que sus contenidos, aún teniendo el mismo propósito, suelen estar
organizados de distintas maneras o directamente no estar presentes. Varias
páginas están organizadas de distintas maneras, algunas usando un esquema más
antiguo y otras uno más nuevo. Por ejemplo, la página de la serie Los
Soprano\footnote{\url{https://tvtropes.org/pmwiki/pmwiki.php/Series/TheSopranos}}
presenta sus \textit{tropos} organizados en carpetas por rango alfabético,
mientras que  en la película El Viaje de
Chihiro\footnote{\url{https://tvtropes.org/pmwiki/pmwiki.php/Anime/SpiritedAway}}
están agrupados en una lista más simple sin carpetas. O por ejemplo, una página
de una obra, que presenta una lista de \textit{tropos}, no tiene la misma
información que una página de \textit{tropos}, que presenta una lista de obras
que utilizan este recurso, muchas veces organizado por géneros o de otras
maneras.

En general, todas las páginas presentan una serie de distintas estructuras para
organizar la misma información, sin embargo, no se sabe de antemano cuál por lo
que es imprescindible tener en cuenta en cada momento cómo es la página que se
está explorando antes de extraer los datos. Existen una serie de formatos
comunes que suelen tomar la mayoría de las páginas y que se identificarán en el
capítulo 5 del trabajo, al hacer el análisis de TvTropes, pero conocer todos y
cada uno de los esqueletos que puede tomar una página como base es imposible, ya
que, existen excepciones y esto es también un problema que debe superar el
\textit{scraper}. Para solucionar estos problemas el \textit{scraper} debe
comprobar si es capaz de entender y extraer la información de la página que está
explorando.

Por último, extraer tanta cantidad de información presenta un reto adicional en
términos de eficiencia en el tiempo de ejecución debido a todo el tiempo que
puede tardar una herramienta en recoger tantos contenidos. No siempre se querrán
extraer todos los contenidos, sin embargo, el \textit{scraper} debe adaptarse
siempre al volumen de datos que va a manejar en cada momento.

Por tanto, el problema principal que busca resolver este trabajo es el de
desarrollar una herramienta eficiente y eficaz mediante la cual cualquier
persona interesada en el estudio de los \textit{tropos} pueda obtener todos los
datos de TvTropes estructurados y preparados para su análisis, pudiendo elegir
qué información concreta quiere sin necesidad de extraerla toda y en qué formato
de datos la quiere representada. Esta herramienta debe tener en cuenta la
naturaleza cambiante tanto de los datos como de la fuente de información de
donde los extrae, y debe poder presentarlos en un formato legible tanto por
humanos como por programas, para poder hacer cualquier tipo de estudio sobre
ellos. Estos datos deberán ser completos, correctos y estar siempre actualizados
para evitar que cualquier tipo de análisis esté desfasado.

A continuación se describen los objetivos que busca alcanzar este proyecto tras
haber resuelto los problemas descritos. Estos objetivos se usarán en la última
sección de este trabajo como medida para comprobar si el proyecto ha finalizado
correctamente.

\section{Objetivos}
Una vez descrito el problema que se quiere resolver y los retos que presenta, se
desglosan los objetivos específicos que tiene este trabajo.

\begin{itemize}
    \item \textbf{OBJ01} Diseñar y desarrollar una solución informática que sea
    capaz de entender, extraer, limpiar y preparar la información de cualquier
    página de \textit{tropos} de TvTropes independientemente de su estructura.
    \item \textbf{OBJ02} La solución será capaz de extraer información de
    metadatos de contenido no estructurado, aparte de los \textit{tropos}, que
    permita poder diferenciar entre obras audiovisuales y relacionarlas con
    fuentes de datos externas.
    \item \textbf{OBJ03} Diseñar y desarrollar una aplicación que pueda
    integrarse fácilmente en flujos de trabajo de ciencia de datos y machine
    learning mediante la cual el usuario pueda elegir qué contenidos quiere
    extraer.
    \item \textbf{OBJ04} La solución entenderá la frecuencia con la que cambian
    los contenidos y la organización de las páginas de TvTropes y actualizará la
    información extraída para poder tener siempre los datos más nuevos que
    existen en la web.
    \item \textbf{OBJ05} La solución desarrollada debe ser rápida y eficiente,
    adaptándose a la gran cantidad de datos que tiene que extraer en cada
    momento.
\end{itemize}

