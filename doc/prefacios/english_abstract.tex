\begin{otherlanguage}{english}

\begin{center}
    {\large\bfseries TropesToGo \\ \textit{Scraping} from TvTropes}\\
\end{center}
\begin{center}
    Student's name\\
\end{center}
\vspace{0.5cm}
\noindent\textbf{Keywords}: \textit{tropes}, \textit{TvTropes},
\textit{narratives}, \textit{scraping}, \textit{crawling}, \textit{open source},
\textit{agile}, \textit{user stories}, \textit{TDD}, \textit{DDD},
\textit{incremental development}, \textit{Go}, \textit{CLI}, \textit{logging},
\textit{CSV},
\textit{JSON}
\vspace{0.7cm}

\noindent\textbf{Abstract}\\

Tropes are recognizable resources o patterns which are of great interest in the
generation of narratives so that authors can convey their ideas to the public in
an effective way. This resources are key in the popularity a work can reach, and
they are present in many media such as cinema, television or videogames. The
importance of tropes raises a great deal of interest in them that is reflected
in the emergence of numerous articles and studies, which are reviewed in this
paper. These studies require a large and complete dataset that links works to
its tropes so that computational or statisical analysis can allow new studies of
the current state of the narrative.

TvTropes is a website which holds the largest source of information on tropes.
It contains a great amount of unstructured data that is only intended to be
readable by humans, and is not organized or prepared to be processed by a
computer. In this paper a solution focused on programmers and researches is
proposed so that they can extract this data according to their needs and
integrate it into data science pipelines that will enable these studies.

Throughout this paper we present an analysis of the TvTropes website and design
and develop a scraping and crawling tool in the Go language that extracts
relevant information from TvTropes work metadata and its associated tropes. The
goal of the solution is to be able to understand, extract, clean and prepare
data from any TvTropes tropes page. It will be able to generate proper datasets,
in formats such as CSV or JSON, and understand the frequency with which TvTropes
contents changes.

To achieve the defined objectives, an incremental agile development process
guided by user stories, Test Driven Development and DDD modeling is described
and carried out. This culminates in a free and open source command line tool
called TropesToGo. With this tool, both programmers and researches can obtain
datasets in different formats with the exact data they need. TropesToGo is a
novel solution as there aren't many available datasets that link tropes to
works, or other tools that can extract this kind of information from TvTropes.

\end{otherlanguage}