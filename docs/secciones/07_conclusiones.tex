\chapter{Conclusiones y trabajo futuro}
Finalmente, en este último capítulo se expresan las conclusiones del trabajo con
relación a los objetivos definidos al principio del mismo y las posibles
ampliaciones que podrían caber de cara al futuro.

\section{Conclusiones}
En relación con los objetivos presentados en el \autoref{chapter:2}, se concluye
lo siguiente:
\begin{itemize}
    \item Se ha conseguido proporcionar un estudio muy amplio sobre un tema con
    poca presencia en la literatura y en internet como es el \textit{scraping}
    de TvTropes. Se han aprovechado diferentes recursos para extraer, limpiar y
    almacenar información de manera ética, siguiendo las condiciones de servicio
    de TvTropes, sus licencias y sin suponer una gran carga en su servidor que
    impidiese su funcionamiento. El haber extraído toda la información de
    \textit{tropos} y obras de múltiples medios audiovisuales cumple la
    \href{https://github.com/jlgallego99/TropesToGo/issues/6}{[HU01]}, mientras
    que las consideraciones relativas a la eficiencia del programa y a no
    impedir el correcto funcionamiento de TvTropes satisfacen la
    \href{https://github.com/jlgallego99/TropesToGo/issues/45}{[HU06]}. 
    \item El seguimiento del desarrollo de forma ágil ha facilitado el adaptarse
    a cambios y dificultades al estar frente a un problema novedoso. Gracias a
    haber llevado a cabo un previo proceso de estudio de los usuarios objetivo,
    planificación de historias de usuario, modelado del problema y análisis de
    la web de TvTropes desde distintos ámbitos, se ha conseguido llevar a su
    completitud un proyecto de \textit{software} libre que cumpla las
    necesidades de sus usuarios.
    \item La herramienta diseñada y desarrollada ofrece de manera libre la
    posibilidad de extraer datos de TvTropes de un modo efectivo y rápido,
    pudiendo mejorar la propuesta de \textit{Tropescraper}. TropesToGo es capaz
    de extraer eficientemente datos de cualquier tipo de medio audiovisual (tal
    y como especifica la
    \href{https://github.com/jlgallego99/TropesToGo/issues/7}{[HU02]}), asociar
    cada obra a una serie de \textit{tropos} de distintos tipos y persistir toda
    esta información en ficheros de distinto formato, limpios y estructurados,
    para su uso en procesos de ciencia de datos, que facilitan la generación de
    nuevos análisis y estudios en torno a narrativas. Los usuarios podrán elegir
    el formato de datos más adecuado para ellos
    (\href{https://github.com/jlgallego99/TropesToGo/issues/30}{[HU05]}), el
    cual está persistido como un fichero fácilmente localizable
    (\href{https://github.com/jlgallego99/TropesToGo/issues/46}{[HU07]}).
    \item La solución ha sido capaz no solo de extraer un mayor número de
    \textit{tropos} por cada obra, asociando aquellos que están distribuidos en
    distintas páginas y subpáginas, sino también información de metadatos
    adicional. Información como el título de la obra, su año o el medio
    audiovisual al que pertenecen permiten a la vez diferenciar obras entre sí y
    poder sacar relaciones conceptuales tanto entre ellas como entre sus
    \textit{tropos}. Esta información de metadatos permite diferenciar entre
    obras con mismo nombre, para evitar ambigüedad y poder usarla en conjunto a
    otras fuentes de datos externas como especifica la
    \href{https://github.com/jlgallego99/TropesToGo/issues/8}{[HU03]}. Además,
    cumple la
    \href{https://github.com/jlgallego99/TropesToGo/issues/57}{[HU08]}, ya que
    se tiene información adicional sobre \textit{tropos}, sus obras y el medio
    al que pertenecen, que permitirán formar relaciones más informadas entre
    ellos.
    \item La herramienta ha sido pensado para ser utilizada en última instancia
    por investigadores, desarrolladores o cualquier persona interesada en los
    flujos de ciencia de datos o estadística. Por tanto, permite extraer los
    datos de forma personalizada, rápidamente y en una sola línea, priorizando
    que los usuarios reciban cuanto antes un conjunto de datos que se adapte a
    sus intereses. Todo ello se ejecuta en un proceso automatizado, lo cual no se
    podría hacer con una aplicación de escritorio o web, y permite poder
    llamarlo dentro de otros programas o en utilidades como \texttt{cron} para
    realizar ejecuciones periódicas sin necesidad de interacción por parte del usuario. 
    \item Adicionalmente, se ha tenido en cuenta que los contenidos de TvTropes
    cambian con una alta frecuencia, siguiendo la
    \href{https://github.com/jlgallego99/TropesToGo/issues/9}{[HU04]}. La
    solución es capaz de gestionar la última fecha de actualización de todos los
    contenidos extraídos e inferir si estos han cambiado en TvTropes, para poder
    actualizar sobre lo que ya se tiene y que los usuarios tengan siempre los
    datos más novedosos posibles, que facilitarán el desarrollo de mejores y más
    novedosos estudios
\end{itemize}

Este trabajo ha supuesto un gran reto personal, debido a que he necesitado de
aprender numerosos conceptos y tecnologías que no había visto hasta ahora
relativos al \textit{scraping}. Planificar y seguir un desarrollo ágil de un
proyecto complejo y amplio como este ha resultado muy satisfactorio, gracias a
todo lo que he podido aprender tanto por mi esfuerzo personal como a la
ayuda del tutor.

Considero que este proyecto es único actualmente, ya que como se ha visto en el
estado del arte no existe prácticamente ninguna solución que aborde este
problema. Hasta ahora no existían muchos conjuntos de datos que recogiesen datos
sobre películas y \textit{tropos}, y mucho menos que lo ampliasen a otros medios
audiovisuales y fuesen capaces de contener tanta información. TropesToGo es
capaz de obtener conjuntos de datos ricos en información que podrán dar a luz a
nuevos estudios en el estudio de los \textit{tropos}. Debido a la complejidad
del \textit{scraping} en sí, y de la distribución de los contenidos en TvTropes,
no es de extrañar que existan pocas opciones a este problema, pero tengo la
certeza de que con este trabajo se han superado muchos de sus retos.

\section{Trabajo futuro}
A lo largo de todo el proyecto, y especialmente en el desarrollo, se ha hecho
mucho hincapié en la complejidad de TvTropes como wiki de información. Una gran
cantidad de usuarios están constantemente actualizando y añadiendo información a
ella, lo que hace que existan muchas páginas desconocidas de las que se podría
sacar bastante valor. Esto hace que la extracción de sus datos pueda ser casi
infinito, razón por la que es imprescindible definir unos límites y haber
seguido una serie de productos mínimamente viables que incrementen la
funcionalidad y su complejidad. Se describen a continuación una serie de ideas y
consideraciones adicionales que, ya sea por falta de tiempo o por complejidad,
no se han desarrollado y quedan como trabajo para el futuro como posibles
incrementos:

\begin{itemize}
    \item Los \textit{tropos} son unos recursos que están siempre dentro de un
    contexto determinado. Existen muchos tipos de ellos, ya sea por género, por
    tipo de narración o por muchos otros aspectos. Esta información adicional
    está contenida en TvTropes, aunque muy distribuida, en distintos índices que
    consideran encasillar cada \textit{tropo} en uno o más tipos. Esto es de
    especial importancia para lanzar análisis que estudien la relación entre
    ellos o la popularidad de \textit{tropos} según el contexto o lugar en el
    que se usen. Por tanto, se podría ampliar la extracción para que tenga esta
    información adicional sobre cada uno de los \textit{tropos} extraídos.
    \item La estrategia que se ha seguido para asociar \textit{tropos} a obras
    es la que seguiría cualquier usuario que navegase por TvTropes: buscar su
    obra preferida y ver cuál su información. Esta dirección de la información,
    de obras a \textit{tropos}, permite encontrar la mayoría de ellos. Sin
    embargo, se ha visto que esto no siempre es del todo cierto; existen
    numerosos casos en los que, dentro de la página de un \textit{tropo}, se
    hace referencia a que aparece en una obra, pero luego en la página de esa
    obra no aparece. Una ampliación del \textit{scraper} debería tener en cuenta
    esto, haciendo el conjunto de datos generado aún más diverso y completo.
    \item La extensa información de TvTropes va más allá y es que, si bien este
    trabajo se ha centrado en la asociación entre \textit{tropos} y obras, en
    las páginas de TvTropes se hace constantemente referencias a obras entre sí,
    estableciendo relaciones complejas entre ellas. Estas relaciones pueden ser
    de pertenencia a una saga o relación temática, entre otras, y pueden dar un
    nuevo ángulo a las relaciones entre ellas y sus \textit{tropos} que pueden
    resultar muy interesantes.
    \item Si bien TropesToGo es capaz de mejorar enormemente la extracción
    realizada por \textit{Tropescraper}, esta herramienta tenía la ventaja de
    que podía pausarse su ejecución en cualquier momento y guardaba una caché
    local en disco de los contenidos extraídos hasta el momento, de forma que en
    sucesivas ejecuciones se podría reanudar sin repetir extracciones ya
    realizadas. Esto tiene sus propios problemas, como la sobrecarga que sería
    hacer tantas escrituras y lecturas en disco, pero como trabajo futuro
    sería muy adecuado buscar una mejor solución a este problema para poder
    interrumpir y reanudar el largo proceso cuando se necesite.
    \item Con respecto a lo anterior, se podría mejorar el principal cuello de
    botella que tiene TropesToGo: las peticiones HTTP. Si bien el programa es
    rápido y eficiente, las esperas entre peticiones alargan mucho el proceso,
    que se podría reducir con el uso de \textit{proxies}. Mediante un único proxy
    se podría tener una caché persistida en el sistema de todas las páginas que
    se hayan pedido, permitiendo no sobrecargar al servidor de TvTropes al
    volver a pedir una misma página en diferentes ejecuciones del programa.
    Además, si se usasen varios \textit{proxies} distribuidos en distintos
    lugares, rotando cada cierto tiempo entre ellos, se podría tener un
    \textit{crawling} que no se tuviese que preocupar de esperas ni de posibles
    denegaciones de acceso al acceder siempre a través de un anfitrión distinto.
\end{itemize}