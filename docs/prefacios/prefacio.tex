\thispagestyle{empty}

\begin{center}
{\large\bfseries TropesToGo \\ \textit{Scraping} de TvTropes }\\
\end{center}
\begin{center}
Jose Luis Gallego Peña\\
\end{center}

%\vspace{0.7cm}

\vspace{0.5cm}
\noindent\textbf{Palabras clave}: \textit{tropos}, \textit{TvTropes},
\textit{narrativa}, \textit{scraping}, \textit{crawling}, \textit{software
libre}, \textit{desarrollo ágil}, \textit{historias de usuario}, \textit{TDD},
\textit{DDD}, \textit{desarrollo incremental}, \textit{Go}, \textit{CLI},
\textit{logging}, \textit{CSV},
\textit{JSON}
\vspace{0.7cm}

\noindent\textbf{Resumen}\\
	
Los \textit{tropos} son unos recursos o patrones reconocibles que tienen un gran
interés en la generación de narrativas para que sus autores puedan transmitir al
público sus ideas de la manera más efectiva posible. Estos recursos son clave en
la popularidad que puede llegar a tener una obra, y están presentes en numerosos
medios audiovisuales como el cine, la televisión o los videojuegos. La
importancia de los \textit{tropos} suscita un gran interés que se ve reflejado
en la aparición de numerosos estudios, los cuales se revisan en este trabajo.
Estos estudios necesitan de un conjunto de datos amplio y completo que relacione
obras y \textit{tropos} para su análisis con técnicas estadísticas o
computacionales que permitan analizar el estado actual de la narrativa.

La web TvTropes es la mayor fuente de información sobre estos recursos. Contiene
una gran cantidad de datos semiestructurados que únicamente están pensados para
ser legibles por humanos y no están organizados ni preparados para su
procesamiento en un ordenador. En este trabajo se ofrece una solución a
investigadores y programadores para que puedan extraer estos datos según las
necesidades que tengan y puedan integrarlos en procesos de ciencia de datos que
propicien estos análisis.

A lo largo del trabajo se presenta un análisis de la web de TvTropes y se diseña
y desarrolla una herramienta de \textit{scraping} y \textit{crawling} en el
lenguaje Go que extrae información relevante de metadatos y \textit{tropos}
asociados de TvTropes. El objetivo de la solución es que sea capaz de entender,
extraer, limpiar y preparar los datos de cualquier página de \textit{tropos} de
TvTropes, genere ficheros de datos adecuados en formatos como CSV y JSON y
entienda la frecuencia con la que cambian los contenidos de TvTropes.

Para alcanzar los objetivos, se describe y lleva a cabo un proceso de desarrollo
ágil incremental guiado por historias de usuario, \textit{Test Driven
Development} y modelado con DDD que culmina en una herramienta de línea de
comandos libre llamada TropesToGo. Con esta herramienta, investigadores y
programadores pueden obtener conjuntos de datos de distinto formato y con la
información que necesitan. TropesToGo constituye una solución novedosa, ya que
hay muy pocos conjuntos de datos disponibles que asocien \textit{tropos} a obras
o herramientas que permitan extraer esta información de TvTropes.

\cleardoublepage

\begin{otherlanguage}{english}

\begin{center}
    {\large\bfseries TropesToGo \\ \textit{Scraping} from TvTropes}\\
\end{center}
\begin{center}
    Student's name\\
\end{center}
\vspace{0.5cm}
\noindent\textbf{Keywords}: \textit{tropes}, \textit{TvTropes},
\textit{narratives}, \textit{scraping}, \textit{crawling}, \textit{open source},
\textit{agile}, \textit{user stories}, \textit{TDD}, \textit{DDD},
\textit{incremental development}, \textit{Go}, \textit{CLI}, \textit{logging},
\textit{CSV},
\textit{JSON}
\vspace{0.7cm}

\noindent\textbf{Abstract}\\

Tropes are recognizable resources o patterns which are of great interest in the
generation of narratives so that authors can convey their ideas to the public in
an effective way. This resources are key in the popularity a work can reach, and
they are present in many media such as cinema, television or videogames. The
importance of tropes raises a great deal of interest in them that is reflected
in the emergence of numerous articles and studies, which are reviewed in this
paper. These studies require a large and complete dataset that links works to
its tropes so that computational or statisical analysis can allow new studies of
the current state of the narrative.

TvTropes is a website which holds the largest source of information on tropes.
It contains a great amount of unstructured data that is only intended to be
readable by humans, and is not organized or prepared to be processed by a
computer. In this paper a solution focused on programmers and researches is
proposed so that they can extract this data according to their needs and
integrate it into data science pipelines that will enable these studies.

Throughout this paper we present an analysis of the TvTropes website and design
and develop a scraping and crawling tool in the Go language that extracts
relevant information from TvTropes work metadata and its associated tropes. The
goal of the solution is to be able to understand, extract, clean and prepare
data from any TvTropes tropes page. It will be able to generate proper datasets,
in formats such as CSV or JSON, and understand the frequency with which TvTropes
contents changes.

To achieve the defined objectives, an incremental agile development process
guided by user stories, Test Driven Development and DDD modeling is described
and carried out. This culminates in a free and open source command line tool
called TropesToGo. With this tool, both programmers and researches can obtain
datasets in different formats with the exact data they need. TropesToGo is a
novel solution as there aren't many available datasets that link tropes to
works, or other tools that can extract this kind of information from TvTropes.

\end{otherlanguage}

\cleardoublepage

\thispagestyle{empty}

\noindent\rule[-1ex]{\textwidth}{2pt}\\[4.5ex]

D. \textbf{Juan Julián Merelo Guervós}, Profesor del departamento de Arquitectura y Tecnología de Computadores 

\vspace{0.5cm}

\textbf{Informo:}

\vspace{0.5cm}

Que el presente trabajo, titulado \textit{\textbf{Scraping de TvTropes}}, ha sido realizado bajo mi supervisión por \textbf{Jose Luis Gallego Peña}, y autorizo la defensa de dicho trabajo ante el tribunal que corresponda.

\vspace{0.5cm}

Y para que conste, expiden y firman el presente informe en Granada a junio de 2023.

\vspace{1cm}

\textbf{El director: }

\vspace{5cm}

\noindent \textbf{(Juan Julián Merelo Guervós)}

\chapter*{Agradecimientos}

A todos mis amigos, tanto los que han estado en Granada a mi lado en largas
tardes de trabajo, como los que, aún estando lejos, han intentado siempre
ayudarme. Vuestro apoyo y ánimos han sido esenciales.

A mis familiares, por el apoyo e interés con mi trabajo y el animarme con mis
estudios.

A mi tutor, por su dedicación e imprescindible ayuda durante todo el desarrollo
del trabajo.

Y un agradecimiento especial a Antonio Sánchez Alcaraz, por proponer el nombre
que da título a este proyecto.